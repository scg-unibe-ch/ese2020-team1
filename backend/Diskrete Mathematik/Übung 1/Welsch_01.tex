\documentclass[12pt,a4paper]{article}
\usepackage[utf8]{inputenc}
\usepackage[german]{babel}
\usepackage[T1]{fontenc}
\usepackage{amsmath}
\usepackage{amsfonts}
\usepackage{amssymb}
\usepackage{graphicx}
\usepackage{lmodern}
\usepackage[left=2cm,right=2cm,top=3cm,bottom=2cm]{geometry}
\usepackage{scrlayer-scrpage}
\usepackage[shortlabels]{enumitem}

\pagestyle{headings}
\ihead{Diskrete Mathematik, HS 2020\\Prof. Christian Cachin}
\ohead{Pascale Welsch\\13-204-821}
\begin{document}
\section*{Übung 1}
\subsection*{1.1 Von der natürlichen Sprache zur Aussagenlogik}
\begin{enumerate}[a)]
\item Definition der Aussagen:
\begin{itemize}
\item e: Die Eintrittsgebühr wurde bezahlt.
\item k: Die Kontaktangaben wurden hinterlegt.
\item c: Der Club darf betreten werden.
\end{itemize}

Aus dem Kontext wird geschlossen, dass jemand, der die Eintrittsgebühr nicht bezahlt oder die Kontaktdaten nicht hinterlegt hat, den Club nicht betreten darf. Es wird deshalb die Biimplikation und nicht die Implikation verwendet.


\begin{tabular}{c|c|c|c|p{0.65\textwidth}}
e & k & c & (e $\land$ k) $\leftrightarrow$ c & Aussage\\
\hline
0 & 0 & 0 & 1 & Jemand, der die Eintrittsgebühr nicht bezahlt und die Kontaktdaten nicht hinterlegt hat, darf nicht eintreten: \textbf{wahr}\\
\hline
0 & 0 & 1 & 0 & Jemand, der die Eintrittsgebühr nicht bezahlt und die Kontaktdaten nicht hinterlegt hat, darf  eintreten: \textbf{falsch}\\
\hline

0 & 1 & 0 & 1 & Jemand, der die Eintrittsgebühr nicht bezahlt und die Kontaktdaten hinterlegt hat, darf nicht eintreten: \textbf{wahr}\\
\hline
0 & 1 & 1 & 0 & Jemand, der die Eintrittsgebühr nicht bezahlt und die Kontaktdaten hinterlegt hat, darf eintreten: \textbf{falsch}\\
\hline
1 & 0 & 0 & 1 & Jemand, der die Eintrittsgebühr bezahlt und die Kontaktdaten nicht hinterlegt hat, darf nicht eintreten: \textbf{wahr}\\
\hline
1 & 0 & 1 & 0 & Jemand, der die Eintrittsgebühr bezahlt und die Kontaktdaten nicht hinterlegt hat, darf eintreten: \textbf{falsch}\\
\hline
1 & 1 & 0 & 0 & Jemand, der die Eintrittsgebühr bezahlt und die Kontaktdaten hinterlegt hat, darf nicht eintreten: \textbf{falsch}\\
\hline
1 & 1 & 1 & 1 & Jemand, der die Eintrittsgebühr bezahlt und die Kontaktdaten hinterlegt hat, darf eintreten:\textbf{ wahr}\\
\end{tabular}

\item Definition der Aussagen:
\begin{itemize}
\item e: Die Eintrittsgebühr wurde bezahlt.
\item k: Die Kontaktangaben wurden hinterlegt.
\item c: Der Club darf betreten werden.
\item z: Die Eintrittsgebühr wird rückerstattet.
\end{itemize}



\begin{tabular}{c|c|c|c|p{0.15\textwidth}|p{0.6\textwidth}}
e & k & c & z & ((e $\land$ k) $\leftrightarrow$ c) $\land$ ((e $\land$ $\neg$k) $\leftrightarrow$ z) & Aussage\\
\hline
0 & 0 & 0 & 0 & 1 & Kein Eintritt bezahlt, keine Kontaktdaten hinterlegt, Club darf nicht betreten werden, keine Rückerstattung: \textbf{wahr}\\
\hline
0 & 0 & 0 & 1 & 0 & Kein Eintritt bezahlt, keine Kontaktdaten hinterlegt, Club darf nicht betreten werden, Rückerstattung: \textbf{falsch}\\
\hline
0 & 0 & 1 & 0 & 0 & Kein Eintritt bezahlt, keine Kontaktdaten hinterlegt, Club darf betreten werden, keine Rückerstattung: \textbf{falsch}\\
\hline
0 & 0 & 1 & 1 & 0 & Kein Eintritt bezahlt, keine Kontaktdaten hinterlegt, Club darf betreten werden, Rückerstattung: \textbf{falsch}\\
\hline
0 & 1 & 0 & 0 & 1 & Kein Eintritt bezahlt, Kontaktdaten hinterlegt, Club darf nicht betreten werden, keine Rückerstattung: \textbf{wahr}\\
\hline
0 & 1 & 0 & 1 & 0 & Kein Eintritt bezahlt, Kontaktdaten hinterlegt, Club darf nicht betreten werden, Rückerstattung: \textbf{falsch}\\
\hline
0 & 1 & 1 & 0 & 0 & Kein Eintritt bezahlt, Kontaktdaten hinterlegt, Club darf betreten werden, keine Rückerstattung: \textbf{falsch}\\
\hline
0 & 1 & 1 & 1 & 0 & Kein Eintritt bezahlt, Kontaktdaten hinterlegt, Club darf betreten werden, Rückerstattung: \textbf{falsch}\\
\hline
1 & 0 & 0 & 0 & 0 & Eintritt bezahlt, keine Kontaktdaten hinterlegt, Club darf nicht betreten werden, keine Rückerstattung: \textbf{falsch}\\
\hline
1 & 0 & 0 & 1 & 1 & Eintritt bezahlt, keine Kontaktdaten hinterlegt, Club darf nicht betreten werden, Rückerstattung: \textbf{wahr}\\
\hline
1 & 0 & 1 & 0 & 0 & Eintritt bezahlt, keine Kontaktdaten hinterlegt, Club darf betreten werden, keine Rückerstattung: \textbf{falsch}\\
\hline
1 & 0 & 1 & 1 & 0 & Eintritt bezahlt, keine Kontaktdaten hinterlegt, Club darf betreten werden, Rückerstattung: \textbf{falsch}\\
\hline
1 & 1 & 0 & 0 & 0 & Eintritt bezahlt, Kontaktdaten hinterlegt, Club darf nicht betreten werden, keine Rückerstattung: \textbf{falsch}\\
\hline
1 & 1 & 0 & 1 &  0 & Eintritt bezahlt, Kontaktdaten hinterlegt, Club darf nicht betreten werden, Rückerstattung: \textbf{falsch}\\
\hline
1 & 1 & 1 & 0 & 1 & Eintritt bezahlt, Kontaktdaten hinterlegt, Club darf betreten werden, keine Rückerstattung: \textbf{wahr}\\
\hline
1 & 1 & 1  & 1 & 0 & Eintritt bezahlt, Kontaktdaten hinterlegt, Club darf betreten werden, Rückerstattung: \textbf{falsch}\\
\end{tabular}

\item Definition der Aussagen:
\begin{itemize}
\item s: Schimmel wächst auf Esswaren
\item l: Man riskiert eine Lebensmittelvergiftung
\item w: Es handelt sich um Weichkäse
\end{itemize}
Man kann auch eine Lebensmittelvergiftung riskieren, wenn kein Schimmel auf den Esswaren wächst. Es wird deshalb die Implikation und nicht die Biimplikation verwendet.


\begin{tabular}{c|c|c|c|p{0.65\textwidth}}
s & l & w & s $\rightarrow$ (l $\oplus$ w) & Aussage\\
\hline
0 & 0 & 0 & 1 & Kein Schimmel auf Esswaren, kein Weichkäse, man riskiert keine Lebensmittelvergiftung: \textbf{Keine Aussage/wahr}\\
\hline
0 & 0 & 1 & 1 & Kein Schimmel auf Esswaren, Weichkäse, man riskiert keine Lebensmittelvergiftung: \textbf{Keine Aussage/wahr}\\
\hline
0 & 1 & 0 & 1 & Kein Schimmel auf Esswaren, kein Weichkäse, man riskiert eine Lebensmittelvergiftung: \textbf{Keine Aussage/wahr}\\
\hline
0 & 1 & 1 & 1 & Kein Schimmel auf Esswaren, Weichkäse, man riskiert eine Lebensmittelvergiftung: \textbf{Keine Aussage/wahr}\\
\hline
1 & 0 & 0 & 0 & Schimmel auf Esswaren, kein Weichkäse, man riskiert keine Lebensmittelvergiftung: \textbf{falsch}\\
\hline
1 & 0 & 1 & 1 & Schimmel auf Esswaren, Weichkäse, man riskiert keine Lebensmittelvergiftung: \textbf{wahr}\\
\hline
1 & 1 & 0 & 1 & Schimmel auf Esswaren, kein Weichkäse, man riskiert eine Lebensmittelvergiftung: \textbf{wahr}\\
\hline
1 & 1 & 1 & 0 & Schimmel auf Esswaren, Weichkäse, man riskiert eine Lebensmittelvergiftung: \textbf{falsch}\\
\end{tabular}

\item Definition der Aussagen:
\begin{itemize}
\item r: In Bern regnet es von September bis Dezember nie
\item ü: Alle Studierenden lösen sämtliche Übungen allein und fehlerfrei
\item p: An der Prüfung erhalten alle eine 6
\end{itemize}
Die Studierenden könnten auch alle eine 6 erhalten, wenn es regnet oder die Übungen nicht fehlerfrei gelöst wurden. Es wird deshalb die Implikation und nicht die Biimplikation verwendet.

\begin{tabular}{c|c|c|c|p{0.65\textwidth}}
r & ü & p & (r $\land$ ü) $\rightarrow$ p & Aussage\\
\hline
0 & 0 & 0 & 1 & In Bern regnet es (irgendwann) zwischen September und Dezember, die Übungen wurden nicht fehlerfrei gelöst, nicht alle erhalten eine 6: \textbf{Keine Aussage/wahr}\\
\hline
0 & 0 & 1 & 1 & In Bern regnet es (irgendwann) zwischen September und Dezember, die Übungen wurden nicht fehlerfrei gelöst, alle erhalten eine 6: \textbf{Keine Aussage/wahr}\\
\hline
0 & 1 & 0 & 1 & In Bern regnet es (irgendwann) zwischen September und Dezember, die Übungen wurden fehlerfrei gelöst, nicht alle erhalten eine 6: \textbf{Keine Aussage/wahr}\\
\hline
0 & 1 & 1 & 1 & In Bern regnet es (irgendwann) zwischen September und Dezember, die Übungen wurden fehlerfrei gelöst, alle erhalten eine 6: \textbf{Keine Aussage/wahr}\\
\hline
1 & 0 & 0 & 1 & In Bern regnet es zwischen September und Dezember nie, die Übungen wurden nicht fehlerfrei gelöst, nicht alle erhalten eine 6: \textbf{Keine Aussage/wahr}\\
\hline
1 & 0 & 1 & 1 &  In Bern regnet es zwischen September und Dezember nie, die Übungen wurden nicht fehlerfrei gelöst, alle erhalten eine 6: \textbf{Keine Aussage/wahr}\\
\hline
1 & 1 & 0 & 0 &  In Bern regnet es zwischen September und Dezember nie, die Übungen wurden fehlerfrei gelöst, nicht alle erhalten eine 6: \textbf{falsch}\\
\hline
1 & 1 & 1 & 1 &  In Bern regnet es zwischen September und Dezember nie, die Übungen wurden fehlerfrei gelöst, alle erhalten eine 6: \textbf{wahr}\\
\end{tabular}
\end{enumerate}

\subsection*{1.2 Wahrheitstabellen}
\begin{enumerate}[a)]
\item 
\begin{tabular}{c|c|c|c}
p & q & r & (p $\lor$ q) $\land$ $\neg$r\\
\hline
0 & 0 & 0 & 0 \\
0 & 0 & 1 & 0 \\
0 & 1 & 0 & 1 \\
0 & 1 & 1 & 0 \\
1 & 0 & 0 & 1 \\
1 & 0 & 1 & 0 \\
1 & 1 & 0 & 1 \\
1 & 1 & 1 & 0 \\
\end{tabular}

\item
\begin{tabular}{c|c|c|c}
p & q & r & (p  $\rightarrow$ (q $\rightarrow$ r))\\
\hline
0 & 0 & 0 & 1\\
0 & 0 & 1 & 1\\
0 & 1 & 0 & 1\\
0 & 1 & 1 & 1\\
1 & 0 & 0 & 1 \\
1 & 0 & 1 & 1\\
1 & 1 & 0 & 0 \\
1 & 1 & 1 & 1\\
\end{tabular}

\item
\begin{tabular}{c|c|c}
p & q & (p  $\rightarrow$ q) $\lor$ ($\neg$ p $\rightarrow$ q))\\
\hline
0 & 0 & 1\\
0 & 1 & 1\\
1 & 0 & 1\\
1 & 1 & 1\\
\end{tabular}

\item
\begin{tabular}{c|c|c}
p & q & (p $\oplus$ q) $\land$ (p $\lor$ $\neg$q)\\
\hline
0 & 0 & 0\\
0 & 1 & 0\\
1 & 0 & 1\\
1 & 1 & 0\\
\end{tabular}
\end{enumerate}

\subsection*{1.3 Tautologien und Kontradiktionen}
\begin{enumerate}[a)]
\item
Um zu zeigen, dass es sich bei einer Aussage um eine Tautologie handelt, reicht es zu zeigen, dass die Aussage bei jeder möglichen Variablenbelegung wahr ist (Wahrheitstabelle).
\begin{enumerate}[1)]
\item
\begin{tabular}{c|c|c|c|c}
p & q & (p $\oplus$ q) & (p $\oplus$ $\neg$q) & (p $\oplus$ q) $\lor$ (p $\oplus$ $\neg$q)\\
\hline
0 & 0 & 0 & 1 & 1\\
0 & 1  & 1 & 0 & 1\\
1 & 0 & 1 & 0 & 1\\
1 & 1 & 0 & 1& 1\\
\end{tabular}

\item
\begin{tabular}{c|c|c|c|c|c|c}
p & q & r & p $\rightarrow$ q & q $\rightarrow$ r & p $\rightarrow$ r & ((p $\rightarrow $ q) $\land$ (q $\rightarrow$ r)) $\rightarrow$ (p $\rightarrow$ r)\\
\hline
0 & 0 & 0 & 1 & 1 & 1 & 1\\
0 & 0 & 1 & 1 & 1 & 1 & 1\\
0 & 1 & 0 & 1 & 0 & 1 & 1\\
0  & 1 & 1 & 1 & 1 & 1 & 1\\
1 & 0 & 0 & 0 & 1 & 0 & 1\\
1 & 0 & 1 & 0 & 1 & 1 & 1\\
1 & 1 & 0 & 1 & 0 & 0 & 1\\
1 & 1 & 1 & 1 & 1 & 1 & 1\\
\end{tabular}
\end{enumerate}
\item Um zu zeigen, dass es sich bei einer Aussage um eine Kontradiktion handelt, reicht es zu zeigen, dass die Aussage bei jeder möglichen Variablenbelegung falsch ist (Wahrheitstabelle).
\begin{enumerate}[1)]
\item
\begin{tabular}{c|c|c|c|c}
p & q & (p $\leftrightarrow$ q) & ($\neg$p $\leftrightarrow$ q) & (p $\leftrightarrow$ q) $\land$ ($\neg$p $\leftrightarrow$ q) \\
\hline
0 & 0 & 1 & 0 & 0\\
0 & 1  & 0 & 1 & 0\\
1 & 0 & 0& 1 & 0\\
1 & 1 & 1 & 0 & 0\\
\end{tabular}


\item
\begin{tabular}{c|c|c|c|c}
p & q & p $\land$ q & (p $\rightarrow$ $\neg$	q) & p $\land $ q $\land$ (p $\rightarrow$ $\neg$q) \\
\hline
0 & 0 & 0 & 1 & 0\\
0 & 1  & 0 & 1 & 0\\
1 & 0 & 0& 1 & 0\\
1 & 1 & 1 & 0 & 0\\
\end{tabular}

\end{enumerate}
\end{enumerate}
\subsection*{1.4 Mehr Tautologien und Kontradiktionen}
\begin{enumerate}[a)]
\item 
\begin{tabular}{c|c|c|c|c}
p & q & p $\land$ q & (p $\oplus$	q) & p $\land$ q $\land$ (p $\oplus$ q) \\
\hline
0 & 0 & 0 & 0 & 0\\
0 & 1  & 0 & 1 & 0\\
1 & 0 & 0& 1 & 0\\
1 & 1 & 1 & 0 & 0\\
\end{tabular}\\
\\
Es handelt sich um eine \textbf{Kontradiktion}.

\item 
\begin{tabular}{c|c|c|c|c|c}
p & q & r & (p $\lor$q) $\land$ ($\neg$p $\lor$ r) & (q $\lor$ r) & (p $\lor$q) $\land$ ($\neg$p $\lor$ r) $\rightarrow$  (q $\lor$ r)\\
\hline
0 & 0 & 0 & 0 $\land$ 1 $\equiv$ 0 & 0 & 1\\
0 & 0 & 1 & 0 $\land$ 1 $\equiv$ 0 & 1 & 1\\
0 & 1 & 0 & 1 $\land$ 1 $\equiv$ 1 & 1 & 1\\
0 & 1 & 1 & 1 $\land$ 1 $\equiv$ 1 & 1 & 1\\
1 & 0 & 0 & 1 $\land$ 0 $\equiv$ 0 & 0 & 1\\
1 & 0 & 1 & 1 $\land$ 1 $\equiv$ 1 & 1 & 1\\
1 & 1 & 0 & 1 $\land$ 0 $\equiv$ 0 & 1 & 1\\
1 & 1 & 1 & 1 $\land$ 1 $\equiv$ 1 & 1 & 1\\
\end{tabular}\\
\\
Es handelt sich um eine \textbf{Tautologie}.

\item 
\begin{tabular}{c|c|c|c}
p & q & s & ((p $\rightarrow$ q)  $\land$ (p $\rightarrow$ s)) $\land$ (p $\rightarrow$ (q $\land$ s))\\
\hline
0 & 0 & 0 & (1 $\land$ 1) $\land$ (0 $\rightarrow$ 0) $\equiv$ 1\\
0 & 0 & 1 &  (1 $\land$ 1) $\land$ (0 $\rightarrow$ 0) $\equiv$ 1\\
0 & 1 & 0 &  (1 $\land$ 1) $\land$ (0 $\rightarrow$ 0) $\equiv$ 1\\
0 & 1 & 1 &  (1 $\land$ 1) $\land$ (0 $\rightarrow$ 1) $\equiv$ 1\\
1 & 0 & 0 &  (0 $\land$ 0) $\land$ (1 $\rightarrow$ 0) $\equiv$ 0\\
1 & 0 & 1 &  (0 $\land$ 1) $\land$ (1 $\rightarrow$ 0) $\equiv$ 0\\
1 & 1 & 0 &  (1 $\land$ 0) $\land$ (1 $\rightarrow$ 0) $\equiv$ 0\\
1 & 1 & 1 &  (1 $\land$ 1) $\land$ (1 $\rightarrow$ 1) $\equiv$ 1\\
\end{tabular}\\
\\
Es handelt sich \textbf{weder um eine Tautologie noch um eine Kontradiktion.}
\item 
\begin{tabular}{c|c|c}
p & q & ($\neg$(p $\leftrightarrow$ q)) $\leftrightarrow$ (p $\leftrightarrow$ $\neg$q)\\
\hline
0 & 0 & $\neg$1 $\leftrightarrow$ 0 $\equiv$ 1\\
0 & 1 & $\neg$0 $\leftrightarrow$ 1 $\equiv$ 1\\
1 & 0 & $\neg$0 $\leftrightarrow$ 1 $\equiv$ 1\\
1 & 1 & $\neg$1 $\leftrightarrow$ 0 $\equiv$ 1\\
\end{tabular}\\
\\
Es handelt sich um eine \textbf{Tautologie}.
\end{enumerate}
\end{document}