\documentclass[12pt,a4paper]{article}
\usepackage[utf8]{inputenc}
\usepackage[german]{babel}
\usepackage[T1]{fontenc}
\usepackage{amsmath}
\usepackage{amsfonts}
\usepackage{amssymb}
\usepackage{graphicx}
\usepackage{lmodern}
\usepackage[left=2cm,right=2cm,top=3cm,bottom=2cm]{geometry}
\usepackage{scrlayer-scrpage}
\usepackage[shortlabels]{enumitem}

\pagestyle{headings}
\ihead{Diskrete Mathematik, HS 2020\\Prof. Christian Cachin}
\ohead{Pascale Welsch\\13-204-821}
\begin{document}
\section*{Übung 3}
\subsection*{3.1 Prädikatenlogische Äquivalenzen}
Damit zwei prädikatenlogische Aussagen äquivalent sind, müssen sie für beliebige Prädikate und Universen dieselben Wahrheitsbelegungen haben. Sind die Aussagen
\begin{enumerate}[I]
\item $\forall x : (P(x) \leftrightarrow Q(x))$
\item $\forall x : P(x) \leftrightarrow \forall x : Q (x)$
\end{enumerate}
äquivalent? Um zu zeigen, dass sie \textbf{nicht} äquivalent sind, genügt es, zwei Prädikate $P(x)$ und $Q(x)$ zu finden, sodass die Wahrheitsbelegungen der Aussagen nicht identisch sind. Zum Beispiel:
\begin{itemize}
\item $P(x):$ x ist gerade.
\item $Q(x):$ x ist ungerade.

\end{itemize}
Man nehme an, das Universum sei $\mathbb{N}$. Mit diesen Prädikaten ist die Aussage I falsch, da eine natürliche Zahl nie gleichzeitig gerade und ungerade sein kann, aber immer entweder das eine oder das andere. Die Aussage II ist wahr, weil sowohl $\forall x: P(x)$ also auch $\forall x: Q(x)$ falsch sind, und wenn zwei Aussagen falsch sind, ist die Biimplikation der beiden Aussagen wahr. Damit sind I und II \textbf{nicht äquivalent}.
\subsection*{3.2 Negation in der Prädikatenlogik}
\begin{enumerate}[a)]
\item \begin{align*}
\neg \forall x \exists y:P(x,y) &\equiv \exists x \neg\exists y:P(x,y)\\
&\equiv \exists x \forall y: \neg P(x,y)
\end{align*}
\item \begin{align*}
\neg \exists y: (Q(y) \land \forall x: R(x,y)) &\equiv \forall y: \neg (Q(y) \land \forall x: R(x,y))\\
&\equiv \forall y: \neg Q(y) \lor \neg \forall x: R(x,y)\\
&\equiv \forall y: \neg Q(y) \lor \exists x: \neg R(x,y)
\end{align*}
\item \begin{align*}
\neg (\forall x \forall y: Q(x,y) \land \exists x \exists y: P(x,y)) &\equiv \neg \forall x \forall y: Q(x,y) \lor \neg \exists x \exists y: P(x,y)\\
&\equiv \exists x \neg\forall y: Q(x,y) \lor \forall x\neg\exists y: P(x,y)\\
&\equiv \exists x\exists y: \neg Q(x,y) \lor \forall x \forall y: \neg P(x,y)
\end{align*}
\item
\begin{align*}
\neg \exists x \exists y: (Q(x,y) \leftrightarrow Q(y,x)) &\equiv \forall x \neg \exists y: (Q(x,y) \leftrightarrow Q(y,x))\\
&\equiv \forall x \forall y : \neg(Q(x,y) \leftrightarrow Q(y,x))\\
&\equiv \forall x \forall y: Q(x,y) \leftrightarrow \neg Q(y,x)
\end{align*}
\end{enumerate}
\subsection*{3.3 Implikationen in der Prädikatenlogik}
Eine Implikation ist nur dann falsch, wenn die Prämisse wahr und die Konklusion falsch ist. Um zu zeigen, dass eine Implikation $p \rightarrow q$ wahr ist, reicht es also zu zeigen, dass, wenn $p$ wahr ist, auch $q$ wahr sein muss.
\begin{enumerate}[a)]
\item $p \rightarrow q$:\\
\\
Wenn $p$ wahr ist, dann gilt für jedes $x$, dass entweder $Q(x)$ oder $P(x)$ oder beide wahr sind.\\
\\
$q$ ist wahr, wenn $\exists x: Q(x)$ oder $\forall x: P(x)$ oder beide wahr sind, bzw. ist $q$ nur dann \textbf{falsch}, wenn $$\neg (\exists x:Q(x) \lor \forall x: P(x)) \equiv \forall x: \neg Q(x) \land \exists x: \neg P(x)$$ \textbf{wahr} ist. $q$ kann also nur dann falsch sein, wenn im Universum \textbf{kein} $x$ existiert, das $Q(x)$ erfüllt \textbf{und mindestens ein}  $x$ existiert, das $P(x)$ nicht erfüllt. In diesem Fall aber müsste $p$ falsch sein, weil es dann mindestens ein $x$ gibt, das \textbf{weder} $Q(x)$ \textbf{noch} $P(x)$ erfüllt. Somit muss $q$ wahr sein, wenn $p$ wahr ist. Damit ist die Implikation $p \rightarrow q$ \textbf{wahr}.
\item $q \rightarrow p$:\\
\\
$q$ ist wahr, wenn $\exists x: Q(x)$ oder $\forall x: P(x)$ oder beide wahr sind.\\
\\
Man nimmt an, dass das Universum mehrere Elemente enthält, von denen \textbf{genau eines} $Q(x)$ erfüllt und \textbf{keines} $P(x)$ erfüllt. In diesem Fall ist $q$ wahr (natürlich kann $q$ auch noch in anderen Fällen wahr sein). Unter den gegebenen Voraussetzungen ist $p$ falsch, weil das Universum Elemente enthält, für die \textbf{weder} $Q(x)$ \textbf{noch} $P(x)$ erfüllt sind. Damit ist die Implikation $q\rightarrow p$ \textbf{falsch}.

\subsection*{Beweismethoden}
\begin{itemize}
\item $P(x,y):$ $x$ und $y$ sind ungerade.
\item$Q(x,y):$ $x\cdot y$ ist ungerade.
\end{itemize}
Zu zeigen: $\forall x \forall y: P(x,y) \rightarrow Q(x,y)$.\\
\\
Oder auch: $U(x)$: $x$ ist ungerade.\\
Zu zeigen: $\forall x \forall y: U(x) \land U(y) \rightarrow U(x\cdot y)$
\paragraph{Direkter Beweis: } $p\rightarrow q$\\
\\
Die Zahlen $c$ und $d$ $\in\mathbb{N}$ sind ungerade, wenn zwei Zahlen $m$ und $n$ $\in\mathbb{N}$ existieren, so dass $c=2\cdot m +1$ und $d=2\cdot n+1$. Wenn man beide multipliziert, erhält man:
\begin{align*}
c\cdot d &= (2m+1)\cdot(2n+1)\\
&=4mn + 2m + 2n + 1\\
&=2\cdot (2mn + m + n) + 1
\end{align*}
Man kann nun definieren, dass $2mn+m+n=z$. $z$ muss ebenfalls eine natürliche Zahl sein, da die natürlichen Zahlen durch Addition und Multiplikation nicht verlassen werden. Damit gilt: $c\cdot d = 2\cdot z +1$ und somit muss $c\cdot d$ ungerade sein.\hfill$\square$
\paragraph{Beweis durch Kontraposition: } $p\rightarrow q \equiv \neg q \rightarrow \neg p$\\
\\
Man nimmt an $c\cdot d$ sei \textbf{nicht} ungerade, also gerade und zeigt, dass dadurch \glqq $c$ \textbf{und} $d$ sind ungerade\grqq{} \textbf{nicht} gilt, also $c$ \textbf{oder} $d$ (oder beide) gerade sind.\\
\\
Wenn $c\cdot d$ gerade ist, so gibt es eine natürliche Zahl $k$, so dass $c\cdot d = 2k \Leftrightarrow k = \frac{c\cdot d}{2}$. $c\cdot d$ ist durch 2 teilbar ($2\mid c\cdot d$), wobei 2 eine Primzahl ist. Das bedeutet, dass entweder $c$ oder $d$ (oder beide) durch zwei teilbar sein müssen ($2\mid c$ oder $2\mid d$).\\
\\
Begründung: \\
Annahme: Wenn 2 kein Teiler von c ist, muss es ein Teiler von d sein. Die einzigen Teiler von 2 sind 2 und 1, damit ist $ggT(2,c)$ entweder 2 oder 1. Da 2 \textbf{kein} Teiler von $c$ ist (gemäss Annahme), gilt $ggT(2,c)=1$, also sind 2 und c \textbf{teilerfremd}. Weil aber $2\mid c\cdot d$ gilt, muss $ggT(2,d)=2$ sein.\hfill$\square$
\\
\paragraph{Beweis durch Kontradiktion: } Um durch zu zeigen, dass eine Implikation $p\rightarrow q$ wahr ist, nimmt man an, dass $p$ und $\neg q$ beide wahr sind und zeigt, dass dies zu einem Widerspruch führt.\\
\\
Annahmen: 
\begin{enumerate}[I]
\item $c = 2m+1$ mit $m\in\mathbb{N}$ und $d = 2n+1$ mit $n\in\mathbb{N}$.
\item $c\cdot d = 2k$ mit $k\in\mathbb{N}$ 
\end{enumerate}
Gestützt auf Annahme I multipliziert man $c$ und $d$
\begin{align*}
c\cdot d &= (2m+1)\cdot(2n+1)\\
&=4mn + 2m + 2n + 1\\
&=2\cdot (2mn + m + n) + 1
\end{align*}
Man kann nun definieren, dass $2mn+m+n=z$. $z$ muss ebenfalls eine natürliche Zahl sein, da die natürlichen Zahlen durch Addition und Multiplikation nicht verlassen werden.\\
$$c \cdot d \overset{(I)}{=} 2\cdot z + 1 = 2 \cdot k \overset{(II)}{=} c\cdot d$$
ist ein Widerspruch, also ist $p\rightarrow q$ wahr.\hfill$\square$

\end{enumerate}
\end{document}